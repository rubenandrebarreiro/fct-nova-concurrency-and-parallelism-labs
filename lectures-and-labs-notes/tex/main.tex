%%%%%%%%%%%%%%%%%%%%%%%%%%%%%%%%%%%%%%%%%
% Uppsala University Assignment Title Page 
% LaTeX Template
% Version 1.0 (27/12/12)
%
% This template has been downloaded from:
% http://www.LaTeXTemplates.com
%
% Original author:
% WikiBooks (http://en.wikibooks.org/wiki/LaTeX/Title_Creation)
% Modified by Elsa Slattegard to fit Uppsala university
% License:
% CC BY-NC-SA 3.0 (http://creativecommons.org/licenses/by-nc-sa/3.0/)

%\title{Title page with logo}
%----------------------------------------------------------------------------------------
%	PACKAGES AND OTHER DOCUMENT CONFIGURATIONS
%----------------------------------------------------------------------------------------

\documentclass[12pt]{article}
\usepackage[english]{babel}
\usepackage[utf8x]{inputenc}
\usepackage{amsmath}
\usepackage{graphicx}
\usepackage{float}
\usepackage{hyperref}
\usepackage[colorinlistoftodos]{todonotes}

\begin{document}

    \begin{titlepage}
    
    \newcommand{\HRule}{\rule{\linewidth}{0.5mm}} % Defines a new command for the horizontal lines, change thickness here
    
    \center % Center everything on the page
     
    %----------------------------------------------------------------------------------------
    %	HEADING SECTIONS
    %----------------------------------------------------------------------------------------
    
    \textsc{\LARGE{Nova School of\\[0.25cm]Science and Technology}}\\[1cm] % Name of your university/college
    \begin{figure}[H]
        \centering
        \begin{minipage}{.5\textwidth}
            \centering
            \includegraphics[width=.8\linewidth]{figures/logos/nova-university-lisbon-logo.png}
        \end{minipage}%
        \begin{minipage}{.5\textwidth}
            \centering
            \vspace{-0.2cm}
            \includegraphics[width=.9\linewidth]{figures/logos/nova-school-science-and-technology-logo.png}
        \end{minipage}
    \end{figure}
    \vspace{0.5cm}
     % Include a department/university logo - this will require the graphicx package
    \textsc{\Large Concurrency and Parallelism}\\[0.5cm] % Major heading such as course name
    \textsc{Master in Computer Science and Engineering}\\[0.5cm] % Minor heading such as course title
    \textsc{\large 2020/2021 - 2\textsuperscript{nd} Semester}\\[0.5cm] % Minor heading such as course title
    
    %----------------------------------------------------------------------------------------
    %	TITLE SECTION
    %----------------------------------------------------------------------------------------
    
    \HRule \\[0.8cm]
        { \huge \bfseries Lecture and Lab Notes}\\[0.4cm] % Title of your document
    \HRule \\[1.5cm]
     
    %----------------------------------------------------------------------------------------
    %	AUTHOR SECTION
    %----------------------------------------------------------------------------------------
    
    \begin{minipage}{0.75\textwidth}
    
        \begin{flushleft} \large
            \textbf{Author(s):}\\
            \begin{itemize}
                \vspace{-0.1cm}
                \item \normalsize{Rúben \textsc{Barreiro} - \href{mailto:r.barreiro@campus.fct.unl.pt}{\emph{r.barreiro@campus.fct.unl.pt}}}
            \end{itemize}
        \end{flushleft}
    
    \end{minipage}\\[2cm]
    
    % If you don't want a supervisor, uncomment the two lines below and remove the section above
    %\Large \emph{Author:}\\
    %John \textsc{Smith}\\[3cm] % Your name
    
    %----------------------------------------------------------------------------------------
    %	DATE SECTION
    %----------------------------------------------------------------------------------------
    
    {\large{Last updated: \today}}\\[2cm] % Date, change the \today to a set date if you want to be precise
    
    \vfill % Fill the rest of the page with whitespace

    \end{titlepage}

    \newpage
    
    
    % ------------- Week 1 -------------
    
    \section*{\Large{\textbf{Lecture 1A - Course Administrivia}} \small{(March 15, 2021)}}
\label{sec:lecture-1A}

\subsection*{\large{\textbf{1A.1. Basic Information of Lecturers}}}
\label{ssec:lecture-1A1}

\noindent The \emph{professor} responsible for the \emph{lectures} and \emph{labs} is:

\begin{itemize}
    \vspace{-0.2cm}
    \item Prof. \emph{João \textsc{Lourenço}} - \href{mailto:joao.lourenco@fct.unl.pt}{\emph{joao.lourenco@fct.unl.pt}}:
    \begin{itemize}
        \vspace{-0.2cm}
        \item \emph{Office Location}:
        \begin{itemize}
            \vspace{-0.1cm}
            \item \emph{Department of Informatics};
            \item Building II - Room P2/9, Ext. 10740;
        \end{itemize}
    \end{itemize}
\end{itemize}

\subsection*{\large{\textbf{1A.2. Discussion Forums}}}
\label{ssec:lecture-1A2}

\noindent There will be available some \emph{discussion forums} such as the following ones:
\begin{itemize}
    \vspace{-0.2cm}
    \item \emph{Piazza}:
    \begin{itemize}
        \vspace{-0.2cm}
        \item \href{piazza.com/fct.unl.pt/spring2021/cp11158/home}{\emph{piazza.com/fct.unl.pt/spring2021/cp11158/home}};
    \end{itemize}
\end{itemize}

\subsection*{\large{\textbf{1A.3. Main Bibliography}}}
\label{ssec:lecture-1A3}

\noindent The \emph{main bibliography} is the following:
\begin{itemize}
    \item \textbf{Structured Parallel Programming: Patterns for Efficient\\Computation}:
    \begin{itemize}
        \vspace{-0.2cm}
        \item \emph{Michael \textsc{McCool}}, \emph{Arch \textsc{Robinson}} and \emph{James \textsc{Reinders}};
        \item \emph{Morgan Kaufmann}, 2012;
        \item \emph{ISBN}: 978-0-12-415993-8;
        \item \href{https://github.com/rubenandrebarreiro/fct-nova-concurrency-and-parallelism-labs/raw/main/books/main-bibliography/structured-parallel-programming-patterns-for-efficient-computation-michael-mccool-arch-robison-and-james-reinders.pdf}{\emph{Click here to download}};
    \end{itemize}
    \item \textbf{Patterns for Parallel Programming}:
    \begin{itemize}
        \vspace{-0.2cm}
        \item \emph{Tim \textsc{Mattson}}, \emph{Beverly \textsc{Sanders}} and \emph{Berna \textsc{Massingill}};
        \item \emph{Addison-Wesley}, 2014;
        \item \emph{ISBN}: 0-321-22811-1;
        \item \href{https://github.com/rubenandrebarreiro/fct-nova-concurrency-and-parallelism-labs/raw/main/books/main-bibliography/patterns-for-parallel-programming-tim-mattson-beverly-sanders-and-berna-massingill.pdf}{\emph{Click here to download}};
    \end{itemize}
    \newpage
    \item \textbf{Concurrent Programming: Algorithms, Principles, and\\Foundations}:
    \begin{itemize}
        \vspace{-0.2cm}
        \item \emph{Michael \textsc{Raynal}};
        \item \emph{Springer-Verlag Berlin Heidelberg}, 2013;
        \item \emph{ISBN}: 978-3-642-32026-2;
        \item \href{https://github.com/rubenandrebarreiro/fct-nova-concurrency-and-parallelism-labs/raw/main/books/main-bibliography/concurrent-programming-algorithms-principles-and-foundations-michel-raynal.pdf}{\emph{Click here to download}};
    \end{itemize}
\end{itemize}

\subsection*{\large{\textbf{1A.4. Additional Bibliography}}}
\label{ssec:lecture-1A4}

\noindent The \emph{additional bibliography} is the following:

\begin{itemize}
    \item \textbf{Programming Concurrency on the JVM: Mastering\\Synchronization, STM, and Actors}:
    \begin{itemize}
        \vspace{-0.2cm}
        \item \emph{Venkat \textsc{Suhramaniam}};
        \item \emph{Pragmatic Bookshelf}, 2011;
        \item \emph{ISBN}: 978-1-934356-76-0;
        \item \href{https://github.com/rubenandrebarreiro/fct-nova-concurrency-and-parallelism-labs/raw/main/books/additional-bibliography/programming-concurrency-on-the-jvm-mastering-synchronization-stm-and-actors-venkat-subramaniam.pdf}{\emph{Click here to download}};
    \end{itemize}
    \item \textbf{The Art of Multiprocessor Programming}:
    \begin{itemize}
        \vspace{-0.2cm}
        \item \emph{Maurice \textsc{Herlihy}}, \emph{Nir \textsc{Shavit}}, \emph{Victor \textsc{Luchangco}} and \emph{Michael \textsc{Spear}};
        \item \emph{Morgan Kauffman}, 2021;
        \item \emph{ISBN}: 978-0-12-415950-1;
        \item \href{https://github.com/rubenandrebarreiro/fct-nova-concurrency-and-parallelism-labs/raw/main/books/additional-bibliography/the-art-of-multiprocessor-programming-maurice-herlihy-nir-shavit-victor-luchangco-and-michael-spear.pdf}{\emph{Click here to download}};
    \end{itemize}
    \item \textbf{Shared-Memory Synchronization}:
    \begin{itemize}
        \vspace{-0.2cm}
        \item \emph{Michael \textsc{Scott}};
        \item \emph{Morgan \& Claypool}, 2013;
        \item \emph{ISBN}: 978-1-608-45956-8;
        \item \href{https://github.com/rubenandrebarreiro/fct-nova-concurrency-and-parallelism-labs/raw/main/books/additional-bibliography/shared-memory-synchronization-michael-scott.pdf}{\emph{Click here to download}};
    \end{itemize}
    \newpage
    \item \textbf{Principles of Concurrent and Distributed Programming}:
    \begin{itemize}
        \vspace{-0.2cm}
        \item \emph{Michael \textsc{Ben-Ari}};
        \item \emph{Pearson}, 2006;
        \item \emph{ISBN}: 978-0-321-31283-9;
        \item \href{https://github.com/rubenandrebarreiro/fct-nova-concurrency-and-parallelism-labs/raw/main/books/additional-bibliography/principles-of-concurrent-and-distributed-programming-algorithms-and-models-mordechai-ben-ari.pdf}{\emph{Click here to download}};
    \end{itemize}
\end{itemize}

\subsection*{\large{\textbf{1A.5. Other Bibliography}}}
\label{ssec:lecture-1A5}

\noindent Other \emph{recommended} \emph{bibliography} is the following:

\begin{itemize}
    \item \textbf{Pro Git}:
    \begin{itemize}
        \vspace{-0.2cm}
        \item \emph{Scott \textsc{Chacon}} and \emph{Scott \textsc{Chacon}};
        \item \emph{Apress}, 2014;
        \item \emph{ISBN}: 978-1-4842-0076-6;
        \item \href{https://github.com/rubenandrebarreiro/fct-nova-concurrency-and-parallelism-labs/raw/main/books/others/pro-git-scott-chacon-and-ben-straub.pdf}{\emph{Click here to download}};
    \end{itemize}
\end{itemize}

\subsection*{\large{\textbf{1A.6. Syllabus}}}
\label{ssec:lecture-1A6}

\noindent The structure of the \emph{course} is described by the following enumerated topics:

\begin{enumerate}
    \item \textbf{Parallel Architectures}:
    \begin{itemize}
        \vspace{-0.2cm}
        \item \emph{Flynn's Taxonomy};
        \item \emph{Performance Theory (including Amdahl's and Gustafson's Laws)};
    \end{itemize}
    \item \textbf{Parallel Programming}:
    \begin{itemize}
        \vspace{-0.2cm}
        \item \emph{The spectrum of high-demanding computational problems};
        \item \emph{Regular and irregular problems};
        \item \emph{Strategies for problem decomposition and their mapping to\\programming patterns};
        \item \emph{The transactional and Map-Reduce models};
    \end{itemize}
    \newpage
    \item \textbf{Concurrency Control and Synchronization}:
    \begin{itemize}
        \vspace{-0.2cm}
        \item \emph{Competition and Collaboration};
        \item \emph{Atomicity};
        \item \emph{Linearization};
        \item \emph{Monitors};
        \item \emph{Locks}
        \item \emph{Semaphores};
        \item \emph{Barriers};
        \item \emph{Producer-Consumer};
        \item \emph{Multi-Reader Single-Writer Locks};
        \item \emph{Futures};
        \item \emph{Concurrency in Practice in Java and C};
    \end{itemize}
    \item \textbf{Safety and Liveness}:
    \begin{itemize}
        \vspace{-0.2cm}
        \item \emph{Safety vs. Liveness};
        \item \emph{Progress};
        \item \emph{Deadlock};
        \item \emph{Deadlock Prevention, Avoidance, Detection, and Recovery}; \item \emph{Livelock};
        \item \emph{Livelock Avoidance};
        \item \emph{Priority Inversion};
        \item \emph{Priority Inheritance};
        \item \emph{Lock-Free Algorithms};
    \end{itemize}
    \item \textbf{The Transactional Model}:
    \begin{itemize}
        \vspace{-0.2cm}
        \item \emph{Composite Operations};
        \item \emph{Transactions (Serializability);
        \item \emph{Optimistic Concurrency Control (OCC)}};
        \item \emph{Transactional Memory};
    \end{itemize}
    \newpage
    \item \textbf{Concurrency without Shared Data}:
    \begin{itemize}
        \vspace{-0.2cm}
        \item \emph{Active Objects};
        \item \emph{Message Passing};
        \item \emph{Actors};
    \end{itemize}
\end{enumerate}

\subsection*{\large{\textbf{1A.7. Evaluation}}}
\label{ssec:lecture-1A7}

\begin{itemize}
    \vspace{-0.2cm}
    \item 60\% - \textbf{2 \emph{Tests} (\emph{Individual} and \emph{Online})} [average $\geq$ 8.5 points];
    \vspace{-0.1cm}
    \item 40\% - \textbf{1 \emph{Project} (Groups of 3 Students)} [grade $\geq$ 8.5 points];
    \vspace{-0.1cm}
    \item 3\% - \textbf{Participation in \emph{Classes' Life Cycle}}:
    \begin{itemize}
        \vspace{-0.2cm}
        \item \emph{Lectures};
        \vspace{-0.1cm}
        \item \emph{Labs};
        \vspace{-0.1cm}
        \item \emph{Piazza};
    \end{itemize}
\end{itemize}

\clearpage

    
    \section*{\Large{\textbf{Lecture 1B - Parallel Programming Models and Architectures}} \small{(March 15, 2021)}}
\label{sec:lecture-1B}

\subsection*{\large{\textbf{1B.1. Introduction}}}
\label{ssec:lecture-1B1}

\noindent All \emph{Classical Computers} are now \textbf{parallel}, specially, the modern ones, which support \textbf{parallelism} in \emph{hardware}, through, at least, one \emph{parallel feature}:
\begin{itemize}
    \vspace{-0.1cm}
    \item \emph{Vector Instructions};
    \vspace{-0.1cm}
    \item \emph{Multi-threaded Cores};
    \vspace{-0.1cm}
    \item \textbf{\emph{Multicore Processors}};
    \vspace{-0.1cm}
    \item \emph{Multiple Processors};
    \vspace{-0.1cm}
    \item \emph{Graphics Engines};
    \vspace{-0.1cm}
    \item \emph{Parallel Co-Processors};
\end{itemize}

\noindent This statement does not apply only to \emph{supercomputers}, but even the \emph{smallest} computers, such as \emph{phones}, support many of these features, where, it is necessary to use explicit \textbf{\emph{parallel programming}} to get the most of them.

The \emph{automatic approaches} to \emph{parallelizing} \emph{serial code} \textbf{\emph{cannot}} deal with the shifts in the \emph{algorithm's structures} required for effective \emph{parallelization}.

A \emph{programmer}, in a \emph{modern computing environment}, should not just take advantage of \emph{processors} with \emph{multi-cores}, but, must be able to write \textbf{\emph{scalable}} \emph{applications}, that can take advantage of \textbf{any} amount of \textbf{\emph{parallel hardware}}.

\vspace{0.35cm}

\noindent The feature of \textbf{\emph{scaling}} requires attention to many factors:
\begin{itemize}
    \vspace{-0.2cm}
    \item \emph{Minimization of Data Movement};
    \vspace{-0.2cm}
    \item \emph{Serial Bottlenecks} (including \emph{Locking});
    \vspace{-0.2cm}
    \item Other forms of \emph{Overhead};
\end{itemize}

\noindent Some \textbf{\emph{Parallel Patterns}} can help with this factors, but ultimately, is a responsibility of the \emph{software developer} to produce a \textbf{\emph{good}} \emph{algorithm design}.

In the recent years, it were also made some progress in other \emph{paradigm} of \emph{computing}, known as, \textbf{\emph{Quantum Computing}}, which is also, \emph{inherently} \textbf{\emph{parallel}}, taking advantage of the \textbf{quantum parallelism} offered on the \emph{atomic} and\emph{sub-atomic} levels of \textbf{\emph{elementary particles}}, such as, \textbf{photons}.

\newpage

\subsection*{\large{\textbf{1B.2. Think Parallel}}}
\label{ssec:lecture-1B2}

\noindent The \textbf{\emph{Parallelism}} is an \emph{intuitive} and \emph{common} \emph{human experience}, where, \emph{programmers} naturally accept the concept of \emph{parallel work} via a \emph{group of workers}, often with \emph{specializations}, in order to achieve a \emph{major goal}.

\vspace{0.35cm}

\noindent Some \emph{important} and \emph{relevant} notions are:
\begin{itemize}
    \vspace{-0.2cm}
    \item \textbf{\emph{Serialization}}:
    \vspace{-0.2cm}
    \begin{itemize}
        \item Act of putting some set of operations into a specific order; 
    \end{itemize}
    \vspace{-0.2cm}
    \item \textbf{\emph{Serial Semantics}}:
    \vspace{-0.2cm}
    \begin{itemize}
        \item \emph{Semantics} used even though the \emph{hardware} was naturally \emph{parallel}; 
    \end{itemize}
    \vspace{-0.2cm}
    \item \textbf{\emph{Serial Illusion}}:
    \vspace{-0.2cm}
    \begin{itemize}
        \item \emph{Mental model} of a \emph{computer} executing operations, \emph{sequentially};
        \item Has the problem of programmers came to depend on it too much;
    \end{itemize}
\end{itemize}

\noindent The \textbf{\textit{Serialization}} has its own \emph{benefits}, such as, if someone reads a piece of \emph{serial code} from top to bottom, will also be able to understand the \emph{temporal order} of operations from the structure of the \emph{source code}.

It helps that modern \emph{programming languages} have evolved to use some \emph{structured} \emph{control flow} to emphasize this aspect of \emph{serial semantics}.

Unless it was intentionally injected \emph{randomness}, the \emph{serial programs} are \textbf{\emph{deterministic}}, doing always the same operations in the same order, giving the \emph{same answer}, every time you run them with the \emph{same inputs}.

The \textbf{\emph{Determinism}} is useful for \emph{debugging}, \emph{verification}, and \emph{testing}, but the \emph{deterministic behavior} is not a natural characteristic of \emph{parallel programs}.

Generally speaking, the \emph{timing} of \emph{task execution} in \emph{parallel programs}, in particular the \emph{relative timing}, is often \textbf{\emph{Non-Deterministic}}.

Given that \emph{parallelism} is necessary for \emph{performance}, it would be useful to find an effective approach to \emph{parallel programming} that retains as many of the benefits of \emph{serialization} as possible, yet is also similar to existing practice.

Some of the known \textbf{\emph{Parallel Patterns}} provide \emph{structure} but they can also avoid the existing \textbf{Non-Determinism}, with a few easily visible exceptions where it is \emph{unavoidable} or \emph{necessary} for a better \emph{performance}.

When eliminating \emph{unnecessary} \emph{serialization}, leading to \emph{poor performance}, the current \emph{programming tools} may have many \textbf{\emph{serial traps}} built into them.

\newpage

\noindent The \textbf{\emph{serial traps}} are constructs that make, often \emph{unnecessary}, their \emph{serial assumptions} and they can also exist in the \emph{design} of \emph{algorithms} and in the abstractions used, to estimate the respective \emph{complexity} and \emph{performance}.

\vspace{0.35cm}

\noindent Often, there are two main steps to \emph{think} and \emph{program} in \emph{parallel}:
\begin{enumerate}
    \vspace{-0.2cm}
    \item Learning to recognize \textbf{\emph{serial traps}};
    \vspace{-0.2cm}
    \item Programming in terms of \textbf{\emph{Parallel Patterns}} that capture their best practices and using \emph{efficient} implementations of these \emph{patterns};
\end{enumerate}

\noindent The most difficult part of learning to \emph{program} in \emph{parallel} is \emph{recognizing} and \emph{avoiding} \textbf{\emph{serial traps}} (i.e., \emph{assumptions} of \emph{serial ordering}).

These assumptions are so \emph{common} that, their existence is unnoticed and most of the \emph{programming languages} \emph{unnecessarily} overconstrain the \emph{execution order}, making the \emph{parallel execution} of their \emph{programs} to be difficult.

\emph{Parallelizing} \emph{serial} \emph{programs} written using \emph{iteration} \emph{constructs}, requires to be \emph{recognized} the \emph{semantics} used and \emph{convert} them to the appropriate \emph{parallel structure} and even better would be to \emph{design} \emph{programs} using them.

\vspace{0.5cm}

\subsection*{\large{\textbf{1B.3. Vocabulary and Notation}}}
\label{ssec:lecture-1B3}

\noindent The two fundamental components of algorithms are \textbf{\emph{tasks}} and \textbf{\emph{data}}, where a \textbf{\emph{task}} operates on \textbf{\emph{data}}, either \emph{modifying} it in place or \emph{creating} new \textbf{\emph{data}}.

In a \emph{parallel computation}, the \emph{multiple tasks} need to be \emph{managed} and \emph{coordinated}, where, in particular, \textbf{\emph{dependencies}} between \textbf{\emph{tasks}} need to be \emph{respected}, regarding their particular \emph{ordering} of \emph{execution}.

\vspace{0.35cm}

\noindent The \textbf{\emph{dependencies}} are often but not always associated with the transfer of \textbf{\emph{data}}, between \textbf{\emph{tasks}} and can be categorised as the following two kinds:
\begin{itemize}
    \vspace{-0.2cm}
    \item \textbf{\emph{Data Dependency}}:
    \begin{itemize}
        \vspace{-0.2cm}
        \item \textbf{\emph{Tasks}} that \textbf{\emph{cannot}} be executed, before some of its required \textbf{\emph{data}} be ready, which can be generated by other \textbf{\emph{tasks}};
    \end{itemize}
    \vspace{-0.2cm}
    \item \textbf{\emph{Control Dependency}}:
    \begin{itemize}
        \vspace{-0.2cm}
        \item \emph{Events} or \emph{side effects}, such as, the ones happening due to some \emph{I/O operations}, which need to be \emph{ordered}, in time;
    \end{itemize}
\end{itemize}

\newpage

\noindent For the \emph{task management}, there is the \textbf{\emph{fork-join}} \emph{pattern}, where can exist:
\begin{itemize}
    \vspace{-0.2cm}
    \item \textbf{Fork Points}:
    \vspace{-0.2cm}
    \begin{itemize}
        \item New \emph{serial} \emph{control flows} are created, by \emph{splitting} an existing \emph{serial} \emph{control flow}, previously executed;
    \end{itemize}
    \vspace{-0.2cm}
    \item \textbf{Join Points}:
    \vspace{-0.2cm}
    \begin{itemize}
        \item Two \emph{separate} executed \emph{serial} \emph{control flows} which are \emph{synchronized}, by \emph{merging} them together in one \emph{serial} \emph{control flow};
    \end{itemize}
\end{itemize}

\noindent In a \emph{single} \emph{serial} \emph{control flow}, the \textbf{\emph{tasks}} are \emph{ordered}, according to the \emph{serial semantics}, and due to the implicit \emph{serial} \emph{control flow} before and after these \emph{points}, the \textbf{\emph{control dependencies}} are also needed between \textbf{\emph{fork}} and \textbf{\emph{join}} \textbf{\emph{points}} and the \textbf{\emph{tasks}} that \emph{precede} and \emph{follow} them, respectively.

\vspace{0.5cm}

\subsection*{\large{\textbf{1B.4. Simple Approaches to Parallelization}}}
\label{ssec:lecture-1B4}

\noindent Some simple approaches for \emph{parallelization} are:
\begin{itemize}
    \item \textbf{\emph{Distributed Memory}}:
    \vspace{-0.2cm}
    \begin{itemize}
        \item \emph{Parallel system} based on \emph{processors} \textbf{\emph{linked}} with a \emph{fast network};
        \item The \emph{processors} \textbf{\emph{communicate}} with each other, via \emph{messages};
    \end{itemize}
    \item \textbf{\emph{Owner Computes}}:
    \vspace{-0.2cm}
    \begin{itemize}
        \item \emph{Distribute} the \textbf{\emph{data}} \emph{elements} to \emph{processors};
        \item Each \emph{processor} \emph{updates} its own \textbf{\emph{data}} \emph{elements};
    \end{itemize}
    \item \textbf{\emph{Single Program, Multiple Data}} (\textbf{\emph{SPMD}}):
    \vspace{-0.2cm}
    \begin{itemize}
        \item All \emph{machines} run the \textbf{\emph{same}} \emph{program} on \emph{independent} \textbf{\emph{data}};
        \item The \emph{dominant} form of \emph{parallel computing};
    \end{itemize}
\end{itemize}

\noindent For \textbf{\emph{Shared Memory}}, the \emph{latency} of \emph{communication} between the \emph{processors} is \emph{low}, while, for the \textbf{\emph{Distributed Systems}}, occurs the opposite, with \emph{high} \emph{latencies} on the \emph{communications} between the \emph{processors}, where this \emph{drawback} can be solved, by keeping some \textbf{\emph{data}} \emph{elements} of other \emph{processors} \emph{locally}, with what is called \textbf{\emph{ghost cells}}, \emph{decreasing} a \emph{latency} of its \emph{communications}.

\newpage

\subsection*{\large{\textbf{1B.5. Flynn's Characterization}}}
\label{ssec:lecture-1B5}

\noindent The \emph{parallel processors} can be divided into categories based on whether they have \emph{multiple} \textbf{\emph{flows of control}}, \emph{multiple} \textbf{\emph{streams of data}}, or on both:
\begin{itemize}
    \vspace{-0.2cm}
    \item \textbf{\emph{Single Instruction, Single Data}} (\textbf{\emph{SISD}}):
    \begin{itemize}
        \vspace{-0.2cm}
        \item \emph{Standard} \emph{non-parallel} \emph{processor}, often referred as \emph{scalar processor};
        \item The \emph{individual} \emph{performance} of \emph{scalar processing} is important, since if it is slow, it can end up \emph{dominating} the \emph{global performance};
    \end{itemize}
    \item \textbf{\emph{Single Instruction, Multiple Data}} (\textbf{\emph{SIMD}}):
    \begin{itemize}
        \vspace{-0.2cm}
        \item \emph{Single} \emph{task} \emph{executing} simultaneously on \emph{multiple} \emph{elements} of \emph{data};
        \item The number of \emph{data elements} in a \emph{SIMD}'s \emph{operation} can vary from a small number, such as the 4 to 16 elements in short \emph{vector instructions}, to thousands, as in \emph{streaming vector processors};
        \item The \emph{SIMD processors} are also known as \emph{array processors}, since consist on an \emph{array} of \emph{functional units} with a \emph{shared controller};
    \end{itemize}
    \item \textbf{\emph{Multiple Instruction, Single Data}} (\textbf{\emph{MISD}}):
    \begin{itemize}
        \vspace{-0.2cm}
        \item Not particularly useful and is not used;
    \end{itemize}
    \item \textbf{\emph{Multiple Instruction, Multiple Data}} (\textbf{\emph{MIMD}}):
    \begin{itemize}
        \vspace{-0.2cm}
        \item Separated \emph{streams} of \textbf{\emph{instructions}} and \textbf{\emph{tasks}}, each one with its own \emph{flow of control}, operating on separated \textbf{\emph{data}} \emph{elements};
        \item Defines \emph{multiple cores} in a \emph{single processor}, \emph{multiple processors} in a \emph{single computer}, and \emph{multiple computers} in a \emph{single cluster};
        \item Represents a \textbf{\emph{heterogeneous computer}}, as \emph{multiple processors} using \emph{different} \emph{architectures}, present in the \emph{same} \emph{computer system};
        \item An example would be a \emph{host processor} and a \emph{co-processor} with \emph{different} sets of \textbf{\emph{instructions}} and \textbf{\emph{tasks}};
    \end{itemize}
\end{itemize}

\newpage

\subsection*{\large{\textbf{1B.7. Software Taxonomies}}}
\label{ssec:lecture-1B7}

\noindent Some commonly used \emph{Software Taxonomies} are the following ones:
\begin{itemize}
    \vspace{-0.2cm}
    \item \textbf{\emph{Data Parallel}} (\textbf{\emph{SIMD}}):
    \vspace{-0.2cm}
    \begin{itemize}
        \item \emph{Parallelism} that is a result of \emph{identical} \emph{operations} being applied \emph{concurrently} on \emph{different} \emph{data elements} (e.g., \emph{matrix algorithms});
        \item Difficult to apply it to some \emph{complex problems};
    \end{itemize}
    \vspace{-0.2cm}
    \item \textbf{\emph{Single Program, Multiple Data}} (\textbf{\emph{SPMD}}):
    \vspace{-0.2cm}
    \begin{itemize}
        \item A \emph{single} \emph{application} is executing across \emph{multiple} \emph{processes/threads} on a \emph{MIMD} \emph{architecture};
        \item Most processes execute the same code but do not work in \emph{lock-step};
        \item \emph{Dominant} form of \emph{parallel programming};
    \end{itemize}
\end{itemize}

\subsection*{\large{\textbf{1B.8. Shared Memory (SM)}}}
\label{ssec:lecture-1B8}

\begin{itemize}
    \item \textbf{\emph{Attributes}}:
    \vspace{-0.2cm}
    \begin{itemize}
        \item Has \emph{global} \emph{memory space};
        \item A \emph{processor} use its own \emph{cache} for a part of \emph{global} \emph{memory space};
        \item The \emph{consistency} of the \emph{cache} is maintained by \emph{hardware};
    \end{itemize}
    \item \textbf{\emph{Advantages}}:
    \vspace{-0.2cm}
    \begin{itemize}
        \item \emph{User-friendly} \emph{programming} techniques (\emph{OpenMP} and \emph{OpenACC});
        \item Has \emph{low} \emph{latency} for \emph{data sharing} between \textbf{\emph{tasks}};
    \end{itemize}
    \item \textbf{\emph{Disadvantages}}:
    \vspace{-0.2cm}
    \begin{itemize}
        \item Has \emph{global} \emph{memory} \emph{space-to-CPU} path may be a \emph{bottleneck};
        \item Has \emph{Non-Uniform Memory Access} (\emph{NUMA});
        \item The \emph{programmer} is responsible for the details of \emph{synchronization};
    \end{itemize}
\end{itemize}

\newpage

\subsection*{\large{\textbf{1B.9. Distributed Memory (DM)}}}
\label{ssec:lecture-1B9}

\begin{itemize}
    \item \textbf{\emph{Attributes}}:
    \vspace{-0.2cm}
    \begin{itemize}
        \item The \emph{memory} is \emph{shared} amongst \emph{processors} via \emph{message passing};
    \end{itemize}
    \item \textbf{\emph{Advantages}}:
    \vspace{-0.2cm}
    \begin{itemize}
        \item The \emph{memory} \emph{scales} based on the number of \emph{processors};
        \item The access to a \emph{processor}'s own \emph{memory} is \emph{fast};
        \item Is \emph{cost effective};
    \end{itemize}
    \item \textbf{\emph{Disadvantages}}:
    \vspace{-0.2cm}
    \begin{itemize}
        \item Is \emph{error prone};
        \item The \emph{programmer} is responsible for the details of \emph{communication};
        \item The \emph{complex} \emph{data structures} may be difficult to \emph{distribute};
    \end{itemize}
\end{itemize}

\subsection*{\large{\textbf{1B.10. Software/Hardware Models}}}
\label{ssec:lecture-1B10}

\noindent The \emph{Software} and \emph{Hardware} models \textbf{\emph{do not}} need to match:

\vspace{-0.2cm}
\begin{itemize}
    \item \textbf{\emph{Distributed Memory}} \emph{software} on \textbf{\emph{Shared Memory}} \emph{hardware}:
    \vspace{-0.2cm}
    \begin{itemize}
        \item \textbf{\emph{Message Passing Interface}} (\textbf{\emph{MPI}}), which is designed for \textbf{\emph{Distributed Memory}} (\textbf{\emph{DM}}) \emph{hardware} but is also available on \textbf{\emph{Shared Memory}} (\textbf{\emph{SM}}) \emph{systems};
    \end{itemize}
    \item \textbf{\emph{Shared Memory}} \emph{software} on \textbf{\emph{Distributed Memory}} \emph{hardware}:
    \vspace{-0.2cm}
    \begin{itemize}
        \item \textbf{\emph{Remote Memory Access}} (\textbf{\emph{RMA}}) included within \textbf{\emph{MPI-3}};
        \item \textbf{\emph{Partitioned Global Address Space}} (\textbf{\emph{PGAS}}) \emph{languages}:
        \begin{itemize}
            \item \textbf{\emph{Unified Parallel C}} (extension to \textbf{ISO C 99});
            \item \textbf{\emph{Coarray Fortran}} (\textbf{Fortran 2008});
        \end{itemize}
    \end{itemize}
\end{itemize}

\newpage

\subsection*{\large{\textbf{1B.11. Difficulties of Parallelization}}}
\label{ssec:lecture-1B11}

\noindent There are some common difficulties on \textbf{\emph{Parallelization}}, such as:

\vspace{-0.2cm}
\begin{itemize}
    \item The \textbf{\emph{serialization}} causes \textbf{\emph{bottlenecks}};
    \item The \textbf{\emph{workload}} is not \emph{distributed};
    \item The \textbf{\emph{debugging}} is \textbf{\emph{hard}};
    \item The \textbf{\emph{serial}} approach \textbf{does not} \textbf{\emph{parallelize}};
\end{itemize}

\subsection*{\large{\textbf{1B.12. Performance}}}

\noindent Often, one of the most common \textbf{\emph{serial traps}} is the habit of discussing the \emph{performance} of the \emph{algorithms}, focusing only on the \emph{minimization} of the \emph{total amount} of \emph{computational work} to be done, in \emph{parallel programming}.

\vspace{0.35cm}

\noindent From this common habit, comes two main problems, such as:
\vspace{-0.2cm}
\begin{itemize}
    \item The \emph{computation} itself, may not be the \textbf{\emph{bottleneck}}, since, the \emph{access} to \emph{memory} or its \emph{communication} may constrain the \emph{performance};
    \vspace{-0.1cm}
    \item The \emph{potential} for \emph{scaling performance} on a \emph{parallel computer} is always constrained by the \emph{algorithm's span}, which is the \emph{execution time} needed for the longest \emph{chain of tasks} that, must be performed \emph{sequentially};
\end{itemize}
    
    \section*{\Large{\textbf{Lab 1 - $\pi$ Approximation by Monte Carlo Method}} \small{(March 18, 2021)}}
\label{sec:lab-1}
    
    % ------------- Week 1 -------------
    
    
\end{document}